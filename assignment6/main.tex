\let\negmedspace\undefined
\let\negthickspace\undefined
\documentclass[journal]{IEEEtran}
\usepackage[a5paper, margin=10mm, onecolumn]{geometry}
%\usepackage{lmodern} % Ensure lmodern is loaded for pdflatex
\usepackage{tfrupee} % Include tfrupee package

\setlength{\headheight}{1cm} % Set the height of the header box
\setlength{\headsep}{0mm}     % Set the distance between the header box and the top of the text

\usepackage{xparse}
\usepackage{gvv-book}
\usepackage{gvv}
\usepackage{cite}
\usepackage{amsmath,amssymb,amsfonts,amsthm}
\usepackage{algorithmic}
\usepackage{graphicx}
\usepackage{textcomp}
\usepackage{xcolor}
\usepackage{txfonts}
\usepackage{listings}
\usepackage{enumitem}
\usepackage{mathtools}
\usepackage{gensymb}
\usepackage{comment}
\usepackage[breaklinks=true]{hyperref}
\usepackage{tkz-euclide} 
\usepackage{listings}
% \usepackage{gvv}                                        
\def\inputGnumericTable{}                                 
\usepackage[latin1]{inputenc}                                
\usepackage{color}                                            
\usepackage{array}                                            
\usepackage{longtable}                                       
\usepackage{calc}                                             
\usepackage{multirow}                                         
\usepackage{hhline}                                           
\usepackage{ifthen}                                           
\usepackage{lscape}
\begin{document}

\bibliographystyle{IEEEtran}
\vspace{3cm}

\title{1.1.9.20}
\author{EE24BTECH11058 - P.Shiny Diavajna}
% \maketitle
% \newpage
% \bigskip
{\let\newpage\relax\maketitle}

\renewcommand{\thefigure}{\theenumi}
\renewcommand{\thetable}{\theenumi}
\setlength{\intextsep}{10pt} % Space between text and floats


\numberwithin{equation}{enumi}
\numberwithin{figure}{enumi}
\renewcommand{\thetable}{\theenumi}


\textbf{Question}:\\
Find a point on the $Y$ axis which is equidistant from the points  $\myvec{5\\-2}$ and $\myvec{-3\\2}.$
\\
\textbf{Solution: }
\begin{table}[h!]    
  \centering
  \begin{tabular}[12pt]{ |c|c|c|}
    \hline
    \textbf{Symbol} & \textbf{Value} & \textbf{Description} \\
    \hline
    $\vec{A}$ & \myvec{5\\-2} & First point\\
    \hline 
    $\vec{B}$ & \myvec{-3\\2} & Second point\\
    \hline
    $\vec{Y}$ & \myvec{0\\$y$} & Point on $Y$-Axis equidistant from A and B\\
    \hline
    \end{tabular}

  \caption{Variables Used}
  \label{tab10.5.3.9.1}
\end{table}\\
\begin{align}
\norm{\vec{A}-\vec{Y}}^2&=\norm{\vec{B}-\vec{Y}}^2\\
	\brak{\vec{A}-\vec{Y}}^\top\brak{\vec{A}-\vec{Y}}&=\brak{\vec{B}-\vec{Y}}^\top\brak{\vec{B}-\vec{Y}}\\
	\vec{{A}^\top}\vec{A}-\vec{B^\top}\vec{B}&=2\brak{\vec{A}^\top-\vec{B}^\top}\brak{\vec{Y}}\\
	16=2\brak{\myvec{8&-4}}\myvec{0\\y}\\
y&=-2
\end{align}
The point on the $Y$ axis which is equidistant to $\vec{A}$ and $\vec{B}$ is \myvec{0\\-2}
\begin{figure}[h!]
   \centering
   \includegraphics[width=0.7\linewidth]{figs/fig.png}
   \caption{Plot of the given points and the bisector}
\end{figure}  
\end{document}


